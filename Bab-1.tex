\chapter{Cara Program}

Tujuan dari buku ini adalah mengajarkan kepada kamu bagaimana berfikir seperti seorang ilmuan komputer. Saya sangat suka bagaimana seorang ilmuan komputer berfikir karena mereka menggabungkan aspek-aspek terbaik dari ilmu Matematika, ilmu Teknik, dan ilmu pengetahuan alam. Sebagaimana halnya Matematikawan, ilmuan komputer menggunakan bahasa formal untuk menyatakan ide khususnya yang berkenaan dengan komputasi. Seperti Insinyur(\textit{Engineer}), ilmuan komputer juga merancang sesuatu, merangkai beberapa komponen kedalam sebuah sistem dan mengeveluasi kelebihan dan kekurangan dari berbagai alternatif solusi. Mirip dengan ilmuan pada umumnya, para ilmuan komputer melakukan observasi tingkah laku dari sistem yang kompleks, membentuk beberap hipotesis dan menguji prediksi yang mereka buat.

Satu-satunya keahlian yang paling penting bagi seorang ilmuan komputer adalah keahlian dalam memecahkan masalah (\textit{problem-solving}). Yang saya maksudkan dengan kemampuan memecahkan masalah adalah kemampuan mereka dalam melakukan formulasi masalah, berfikir secara kreatif mengenai solusi dari masalah yang telah diformulasikan dan mengekspresikan sebuah solusi secara jelas dan akurat.  Dan ternyata proses yang kamu lalui ketika belajar program komputer  adalah sebuah kesempatan yang istimewa  dalam berlatih keahlian memecahkan masalah. Itu kenapa judul dari bab ini adalah "Cara Program".

Pada satu sisi kamu akan belajar pemrograman, yang mana merupkah keahlian yang sangat penting. Disisi lainya, kamu akan menggunakan pemrograman sebagai sarana untuk belajar 

\section{Apa itu bahasa pemrograman?}

Bahasa pemrograman yang akan kamu \textbf pelajari adalah Java, yang termasuk sebuah bahasa pemrograman yang relatif baru (Dirilis pertama kali oleh Sun Microsystem pada may 1995). Java adalah salah satu contoh dari bahasa pemrograman level tinggi (\textit{high-level language}); bahasa pemrograman lain yang juga termasuk kategori bahasa pemrograman level tinggi adalah bahasa Python, C atau C++ dan Perl.

Selain istilah bahasa pemrograman level tinggi, terdapat juga istilah bahasa pemrograman level rendah (\textit{low level languages}) dan terkadang dikenal juga dengan istilah bahasa mesin atau bahasa \textit{assembly}. Pada kenyataanya, komputer hanya bisa memahami bahasa pemrograman level rendah. Oleh sebab itu, sebuah program yang ditulis menggunakan bahasa level tinggi harus diterjemahkan terlebih dahulu ke bentuk bahasa level rendah sebelum program tersebut dijalankan. Proses penterjemahan ini membutuhkan waktu sebelum bisa dijalankan oleh komputer, hal ini menjadi salah satu kekurangan dari bahasa pemrograman level tinggi.

Keunggulan dari bahasa  level-tinggi cukup banyak jika dibandingkan dengan kekuranganna. Pertama, jauh lebih mudah untuk membuat program dengan menggunakan bahasa level-tinggi; waktu yang dibutuhkan untuk menuliskan program jauh lebih singkat, penulisannya juga jauh lebih pendek dan mudah dibaca jika dibandingkan dengan bahasa level-rendah. Keuntungan yang kedua adalah portabilitas dalam menjalankannya diberbagai macam arsitektur komputer dengan tanpa modifikasi. Berbeda halnya dengan program yang ditulis dengan bahasa level-rendah yang hanya bisa dijalankan di komputer tertentu, sehingga perlu dimodifikasi jika ingin dijalankan pada komputer dengan arsitektur yang berbeda.

Oleh karena kelebiha-kelebihan tersebut, maka hampir semua program ditulis dengan menggunakan bahasa pemrograman level-tinggi. Bahasa level-rendah hanya digunakan untuk membuat program-program tertentu saja yang jumlahnya juga sedikit.

Terdapat dua cara untuk menterjemahkan sebuah program; \textbf{interpretasi} (\textit{interpreting}) dan \textbf{kompilasi} (\textit{compiling}). Sebuah \textit{interpreter} adalah sebuah program yang membaca sebuah program level-tinggi dan melakukan apa yang diminta oleh program tersebut. Sebagai akibatnya, interpreter akan menterjemahkan program baris demi baris
Sementara \textit{complier} adalah sebuah program yang membaca sebuah program level-tinggi dan menterjemahkan keseluruhan program secara langsung, sebelum menjalankan perintah apa pun dari program tersebut. Sering kali, kamu akan melakukan proses kompilasi secara terpisah terlebih dahulu, kemudian baru  menjalankan program. Pada kasus ini, program level-tinggi disebut dengan istilah kode sumber (\textit{source code}) dan program yang telah diterjemahkan disebut dengan istilah kode objek (\textit{object code}) atau \textit{executable}.


